
% Copyright 2004 by Till Tantau <tantau@users.sourceforge.net>.
%
% In principle, this file can be redistributed and/or modified under
% the terms of the GNU Public License, version 2.
%
% However, this file is supposed to be a template to be modified
% for your own needs. For this reason, if you use this file as a
% template and not specifically distribute it as part of a another
% package/program, I grant the extra permission to freely copy and
% modify this file as you see fit and even to delete this copyright
% notice. 

\documentclass{beamer}
\usepackage[utf8]{inputenc}
\usepackage{graphicx}
\graphicspath{ {./image/} }

% There are many different themes available for Beamer. A comprehensive
% list with examples is given here:
% http://deic.uab.es/~iblanes/beamer_gallery/index_by_theme.html
% You can uncomment the themes below if you would like to use a different
% one:
%\usetheme{AnnArbor}
%\usetheme{Antibes}
%\usetheme{Bergen}
%\usetheme{Berkeley}
%\usetheme{Berlin}
%\usetheme{Boadilla}
%\usetheme{boxes}
%\usetheme{CambridgeUS}
%\usetheme{Copenhagen}
%\usetheme{Darmstadt}
%\usetheme{default}
%\usetheme{Frankfurt}
%\usetheme{Goettingen}
%\usetheme{Hannover}
\usetheme{Ilmenau}
%\usetheme{JuanLesPins}
%\usetheme{Luebeck}
%\usetheme{Madrid}
%\usetheme{Malmoe}
%\usetheme{Marburg}
%\usetheme{Montpellier}
%\usetheme{PaloAlto}
%\usetheme{Pittsburgh}
%\usetheme{Rochester}
%\usetheme{Singapore}
%\usetheme{Szeged}
%\usetheme{Warsaw}

\title{Projet PROG6}

% A subtitle is optional and this may be deleted
\subtitle{Pingouins}

\author{A.~Castel \and C.~Eymond-Laritaz \and G.~Sorin \and L.~Soret \and T.~Vandendorpe \and P.~Reboul}
% - Give the names in the same order as the appear in the paper.
% - Use the \inst{?} command only if the authors have different
%   affiliation.

\institute[Université Grenoble-Alpes] % (optional, but mostly needed)
{
  UFR IM²AG\\
  Université Grenoble-Alpes
}
% - Use the \inst command only if there are several affiliations.
% - Keep it simple, no one is interested in your street address.

\date{Vendredi 1 Juin 2018}
% - Either use conference name or its abbreviation.
% - Not really informative to the audience, more for people (including
%   yourself) who are reading the slides online

% logo of my university
\titlegraphic{
 {\centering
 \vspace{-1.5cm}
 \includegraphics[height=1cm]{logoUnQuart}\hspace*{6.5cm}~%
 \includegraphics[height=1cm]{logo_im2ag}
 }
}

\subject{Informatique}
% This is only inserted into the PDF information catalog. Can be left
% out. 

% If you have a file called "university-logo-filename.xxx", where xxx
% is a graphic format that can be processed by latex or pdflatex,
% resp., then you can add a logo as follows:

%\pgfdeclareimage[height=0.7cm]{team-logo}{logo_team.png}
%\logo{\pgfuseimage{team-logo}}

% Delete this, if you do not want the table of contents to pop up at
% the beginning of each subsection:
\AtBeginSubsection[]
{
  \begin{frame}<beamer>{}
    \tableofcontents[currentsection,currentsubsection]
  \end{frame}
}

% Let's get started
\begin{document}

\begin{frame}
  \titlepage
\end{frame}

\begin{frame}{Outline}
  \tableofcontents
  % You might wish to add the option [pausesections]
\end{frame}

% Section and subsections will appear in the presentation overview
% and table of contents.
\section{IHM}

\subsection{Menu}

\begin{frame}{}{Optional Subtitle}
TODO
\end{frame}

\subsection{Scene de jeu}

% You can reveal the parts of a slide one at a time
% with the \pause command:
\begin{frame}{Le plateau de jeu}
  \begin{block}{Identification des besoins}
    \begin{itemize}
    \item <1-> Information sur le plateau (pingouins, cases détruites, ...)
    \item <2-> Information sur l'état courant des joueurs (joueur courant, scores, ...)
    \item <3-> Que doit faire le joueur?
    \item <4-> Que peut faire le joueur?
    \end{itemize}
  \end{block}
\end{frame}

\begin{frame}{}
  \begin{block}{Autres options}
    \begin{itemize}
    \item <1-> undo/redo
    \item <2-> suggestion
    \item <3-> Autres options (sauvegarder, charger, quitter, ...)
    \end{itemize}
  \end{block}
\end{frame}

\section{Aspects techniques}

%Tests JUnit

\subsection{Moteur}

\begin{frame}{Structure}
\begin{block}{}
\begin{itemize}
\item<1-> Automate à états fini détérministe privées
\item<2-> Données privées sur l'état courant
\item<3-> Fonctions publiques qui modifients l'état selon les règles du jeu
\end{itemize}
\end{block}
\end{frame}

\begin{frame}{Automate à états fini détérministe}
\includegraphics[scale=0.4]{AFD}
\end{frame}

\subsection{Génération de terrains}

\begin{frame}{}
\begin{block}{Choix possibles}
\begin{itemize}
\item<1-> Taille du terrain
\item<2-> Proportion de cases à 1,2 et 3 pingouins
\item<3-> Chargement terrains dans un format prédéfinis
\item<4-> Génération paramétrée
\end{itemize}
\end{block}
\end{frame}

\subsection{fonctionnalités}
\begin{frame}{}
\begin{block}<1->{Défaire/Refaire} %C'est l'undo-rados
Une classe Move pour stocker un coup \newline
Deux piles: 
\begin{itemize}
 \item historique des coups faits (défaire)
 \item historique des coups annulés (refaire)
\end{itemize}
\end{block}
\begin{block}<2->{Sauvegarde/Chargement}
\begin{itemize}
\item Complet
\item Non modifiables (donnée sérialisée)
\end{itemize}
\end{block}
\end{frame}

\begin{frame}{}
\begin{block}{autres}
\begin{itemize}
\item<1-> Intelligence artificielle sur un thread à part
\item<2-> Tests JUnit
\end{itemize}
\end{block}
\end{frame}


\section{IA}

\subsection{Placement en début de partie}
\begin{frame}{}
\begin{block}{}
\begin{itemize}
 \item<1-> le plus proche possible des bancs de poissons
 \item<2-> en evitant les bords autant que p
 \item<3-> en essayant de bloquer l'ennemi autant que possible
 \item<4-> en essayant de ne pas coller tout ses poissons
\end{itemize}
\end{block}
\end{frame}

\begin{frame}{}
\begin{block}{Placement en début de partie}
\begin{itemize}
 \item<1->\includegraphics[scale=0.2]{IA1}
 \item<2->\includegraphics[scale=0.2]{IA2}
\end{itemize}
\end{block}
\end{frame}

\begin{frame}{}
\begin{block}{Spécificité de notre MinMax}
\begin{itemize}
 \item{
  \includegraphics[scale=0.4]{IA4}}
\end{itemize}
\end{block}
\end{frame}


\subsection{Evaluation des coups à venir}
\begin{frame}{}
\begin{block}{}
\begin{itemize}
 \item<1-> Bloquer un pingouin ennemi
 \item<2-> \includegraphics[scale=0.1]{IA5}
 \item<3-> \includegraphics[scale=0.1]{IA6}
\end{itemize}
\end{block}
\end{frame}

\begin{frame}{}
\begin{block}{}
\begin{itemize}
 \item<1-> Ne pas se faire bloquer un pingouin
 \item<2-> Ne pas condamner un pingouin
 \item<3-> \includegraphics[scale=0.09]{IA7}
 \item<4-> \includegraphics[scale=0.09]{IA8}
\end{itemize}
\end{block}
\end{frame}

\begin{frame}{}
\begin{block}{Evaluation des feuilles}
\begin{itemize}
 \item<1-> On essaie de ne pas laisser de trop grande partie de banquise à un pingouin ennemi seul
 \item<2-> On essai de s'isoler sur une grande partie de banquise
 \item<3-> On évite de laisser des grandes îles sans aucun Pingouins dessus.
 \item<4-> Si la partie est finie et que notre score est le plus élevé, on renvoie 
une heuristique maximale.
 \item<5-> Si un ou plusieurs pingouins ennemis sont isolés sur une île suffisamment grande pour s'assurer la victoire, on renvoie une heuristique minimale.
\end{itemize}
\end{block}
\end{frame}

\begin{frame}{}
\begin{block}{}
\begin{itemize}
 \item{ \includegraphics[scale=0.4]{IA9}}
\end{itemize}
\end{block}
\end{frame}

\begin{frame}{}
\begin{block}{Fin de partie}
\begin{itemize}
  \item{On fait un parcours Eulérien}
\end{itemize}
\end{block}
\end{frame}


\subsection{Améliorations}
\begin{frame}{}
\begin{block}{Améliorations possibles}
\begin{itemize}
 \item<1-> Multi-threader le calcul de l’arbre
 \item<2-> Faire en sorte que l’IA difficile puisse changer de stratégie selon la situation
 \item<3-> Optimiser le calcul du parcours Eulérien
\end{itemize}
\end{block}
\begin{block}{Améliorations testées et abandonnées}
\begin{itemize}
 \item<1-> Multi-threader le calcul de l’arbre
 \item<2-> Faire en sorte que l’IA difficile puisse changer de stratégie selon la situation
 \item<3-> Calculer la profondeur a calculer de façon dynamique, pour chaque branches
\end{itemize}
\end{block}
\end{frame}

\subsection{Testes}
\begin{frame}{}
\begin{block}{}
\begin{itemize}
 \item<1-> \includegraphics[scale=0.2]{IA10}
 \item<2-> \includegraphics[scale=0.2]{IA11}
\end{itemize}
\end{block}
\end{frame}

\end{document}


